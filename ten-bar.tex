\documentclass[12pt, a4paper]{article}

\usepackage{xeCJK}
\usepackage[margin=2.0cm]{geometry}
\usepackage{fancyhdr}
\usepackage{graphicx}
\usepackage{amsmath, amssymb, amsfonts}
\usepackage{listings} % code showing
\usepackage{pagecolor} % for code color define
\usepackage{tocloft}

% -- 字型
% -- https://www.overleaf.com/help/193#!CJK
% -- https://www.ffonts.net/cwTeXKai.font.download
\setCJKmainfont{cwTeXKai}

% -- 段落格式
\setlength{\headheight}{28pt}
\pagestyle{fancy}
\fancyhf{}
\setlength\parindent{0pt}
\setlength{\parskip}{0.5em}
\setcounter{secnumdepth}{-1}
\renewcommand{\cftsecleader}{\cftdotfill{\cftdotsep}}

% -- 圖, 表標題使用中文取代
\renewcommand{\figurename}{圖}
\renewcommand{\tablename}{表}

% -- 圖、表標題與圖表之間距離
\setlength{\abovecaptionskip}{10pt}
\setlength{\belowcaptionskip}{10pt}

% \renewcommand{\section}[2]{}

% -- 顯示程式碼格式
\definecolor{codegreen}{rgb}{0,0.6,0}
\definecolor{codegray}{rgb}{0.5,0.5,0.5}
\definecolor{codepurple}{rgb}{0.58,0,0.82}
\definecolor{backcolour}{rgb}{0.95,0.95,0.92}
\lstdefinestyle{pystyle}{
    backgroundcolor=\color{backcolour},
    commentstyle=\color{codegreen},
    keywordstyle=\color{magenta},
    numberstyle=\footnotesize\color{codegray},
    stringstyle=\color{codepurple},
    basicstyle=\ttfamily\footnotesize,
    breakatwhitespace=false,
    breaklines=true,
    captionpos=b,
    keepspaces=true,
    numbers=left,
    numbersep=5pt,
    showspaces=false,
    showstringspaces=false,
    showtabs=false,
    tabsize=2,
    extendedchars=false
}
\lstset{style=pystyle}

% ---------------------------------------------

\begin{document}

\chead{\textbf{2020暑假作業}}
\lhead{SOLab}
\rhead{簡昱凡}
\cfoot{\thepage}

% \ ~ \

\thispagestyle{fancy}

\section{Ten-bar Truss}

十桿衍架為典型衍架結構之一,此篇報告在特定條件下,利用Matlab對該結構進行有限元素及最佳化分析,進而得到能使整體結構質量最小的最佳桿件半徑。
\begin{figure}[!h] 
\includegraphics[scale=0.8]{ten_bars.png} 
\centering
\caption[es]{Ten-bar}
\label{ES}
\end{figure}


\subsection*{Solution}
首先建立函數v2以計算出結構的剛性矩陣、應力及位移。
\lstinputlisting[language=Matlab, caption=有限元素計算, label=code:dwmo]{./v2.m}

將所得的矩陣帶入函數nonlcon,以獲得進行最佳化所需的非線性條件。
\lstinputlisting[language=Matlab, caption=設立非線性條件, label=code:dwmo]{./nonlcon.m}

最後建立目標函數,透過fmincon計算出桿件的最佳半徑。
\lstinputlisting[language=Matlab, firstline=3, lastline=3, caption=目標函數]{./object.m}
\lstinputlisting[language=Matlab, caption=進行最佳化, label=code:dwmo]{./main.m}
運算得到最佳值r=[0.3,0.2663],最佳解f=212410

\end{document}